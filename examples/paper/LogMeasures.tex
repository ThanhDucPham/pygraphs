\documentclass{article}
\usepackage[T2A]{fontenc}
\usepackage[utf8]{inputenc}
\usepackage[english, russian]{babel}

\usepackage{geometry}
\geometry{a4paper, top=1cm, bottom=2cm, left=1cm, right=1cm}

\usepackage{indentfirst}
\usepackage{amssymb,amsfonts,amsmath}
\usepackage{cite,enumerate}
\usepackage[pdftex,colorlinks,unicode,bookmarks]{hyperref}

\usepackage{floatrow}
\usepackage{caption}
\usepackage{subcaption}
\usepackage{multirow}
\usepackage[table]{xcolor}
\usepackage{graphicx}
\usepackage{booktabs}
\usepackage{bm}

\title{
        Воспроизведение результатов статьи в \href{https://github.com/illusionww/py_graphs}{py\_graphs}.
}
\author{Владимир Ивашкин}

\begin{document}

\maketitle

\section{Introduction}

\section{Logarithmic vs. plain measures}
\begin{figure}[H]
	\includegraphics[width=.99\linewidth]{pictures/2_1.png}
	\caption{\label{f_vs1}Logarithmic vs. plain measures for $G(100, (2)0.2, 0.05)$}
\end{figure}

\begin{figure}[H]
	\includegraphics[width=.99\linewidth]{pictures/2_2.png}
	\caption{\label{f_vs2}Logarithmic vs. plain measures for $G(100, (3)0.3, 0.1)$}
\end{figure}

\begin{figure}[H]
	\includegraphics[width=.99\linewidth]{pictures/2_3.png}
	\caption{\label{f_vs3}Logarithmic vs. plain measures for $G(200, (2)0.3, 0.1)$}
\end{figure}
\nopagebreak


\newpage
\section{Competition by Copeland's score}
\begin{table}[H]{\small
	\begin{minipage}{.49\textwidth}		
		\centering
		\begin{tabular}{lrrrrrrrrr}
			\toprule
			\multicolumn{1}{r}{\textbf{Nodes}}         & 100&    100& 100&  100& 200&  200& 200&  200&{\textbf{Sum}} \\
			\multicolumn{1}{r}{\textbf{Classes}}	   &   2&      2&   4&    4&   2&    2&   4&    4&{\textbf{  of}} \\  %\multicolumn{1}{l}
			\multicolumn{1}{r}{$\bm{p_{\mathbf{out}}}$}& 0.1&	0.15& 0.1& 0.15& 0.1& 0.15& 0.1& 0.15&{\textbf{scores}} \\
			\midrule
            logComm &	383 &	547 &	476 &	-66 &	301 &	565 &	592 &	 &	$\bm{2798}$\\
            Comm &	249 &	150 &	308 &	418 &	291 &	212 &	325 &	 &	$\bm{1953}$\\
            SCCT &	316 &	299 &	166 &	44 &	293 &	392 &	412 &	 &	$\bm{1922}$\\
            logHeat &	308 &	314 &	180 &	-264 &	301 &	321 &	343 &	 &	$\bm{1503}$\\
            pWalk &	-81 &	26 &	56 &	418 &	-105 &	-155 &	6 &	 &	$\bm{165}$\\
            SCT &	-74 &	36 &	78 &	44 &	47 &	-24 &	44 &	 &	$\bm{151}$\\
            Heat &	221 &	-342 &	-456 &	418 &	295 &	205 &	-478 &	 &	$\bm{-137}$\\
            RSP &	-96 &	4 &	62 &	-272 &	-32 &	-85 &	-30 &	 &	$\bm{-449}$\\
            Walk &	-90 &	-26 &	78 &	-222 &	-149 &	-125 &	-26 &	 &	$\bm{-560}$\\
            logFor &	-92 &	-44 &	-24 &	-264 &	-63 &	-92 &	-32 &	 &	$\bm{-611}$\\
            FE &	-202 &	-64 &	-44 &	-224 &	-135 &	-169 &	-134 &	 &	$\bm{-972}$\\
            For &	-387 &	-566 &	-456 &	418 &	-525 &	-574 &	-478 &	 &	$\bm{-2568}$\\
            SP-CT &	-455 &	-334 &	-424 &	-448 &	-519 &	-471 &	-544 &	 &	$\bm{-3195}$\\
			\bottomrule
		\end{tabular}
		\\[8pt]\centerline{(a) optimal parameters}
	\end{minipage}\ \ \ \ \
	\begin{minipage}{.49\textwidth}
		\centering
		\begin{tabular}{lrrrrrrrrr}
			\toprule
			\multicolumn{1}{r}{\textbf{Nodes}}         & 100&    100& 100&  100& 200&  200& 200&  200&{\textbf{Sum}}\\
			\multicolumn{1}{r}{\textbf{Classes}}	   &   2&      2&   4&    4&   2&    2&   4&    4&{\textbf{  of}}\\ %\multicolumn{1}{l}
			\multicolumn{1}{r}{$\bm{p_{\mathbf{out}}}$}& 0.1&	0.15& 0.1& 0.15& 0.1& 0.15& 0.1& 0.15&{\textbf{scores}}\\
			\midrule
            logComm &	413 &	568 &	448 &	356 &	332 &	568 &	598 &	598 &	$\bm{3881}$\\
            SCCT &	269 &	274 &	136 &	78 &	340 &	391 &	423 &	360 &	$\bm{2271}$\\
            logHeat &	318 &	183 &	290 &	142 &	340 &	273 &	202 &	98 &	$\bm{1846}$\\
            Comm &	168 &	151 &	222 &	172 &	286 &	258 &	333 &	178 &	$\bm{1768}$\\
            SCT &	58 &	92 &	46 &	90 &	26 &	45 &	38 &	104 &	$\bm{499}$\\
            logFor &	-114 &	60 &	56 &	110 &	-55 &	-115 &	4 &	88 &	$\bm{34}$\\
            Walk &	-84 &	-10 &	132 &	86 &	-140 &	-85 &	30 &	66 &	$\bm{-5}$\\
            pWalk &	-125 &	-40 &	54 &	74 &	-163 &	-79 &	-2 &	-14 &	$\bm{-295}$\\
            FE &	-198 &	-27 &	-27 &	120 &	-120 &	-186 &	-66 &	32 &	$\bm{-472}$\\
            RSP &	-151 &	-1 &	-8 &	78 &	-138 &	-179 &	-106 &	-16 &	$\bm{-521}$\\
            Heat &	299 &	-341 &	-502 &	-490 &	340 &	154 &	-417 &	-515 &	$\bm{-1472}$\\
            SP-CT &	-463 &	-345 &	-320 &	-228 &	-558 &	-462 &	-446 &	-396 &	$\bm{-3218}$\\
            For &	-390 &	-564 &	-588 &	-588 &	-490 &	-583 &	-591 &	-583 &	$\bm{-4377}$\\
			\bottomrule
		\end{tabular}
		\\[8pt]\centerline{(b) 90th percentiles}
	\end{minipage}
	\caption{\label{t_CopComp}Copeland's scores of the measure families on random graphs}
}\end{table}


\newpage
\section{Reject curves}
\begin{table}[H]
	\begin{tabular}{lrrrr}
		\toprule
		Measure & $G(100, (2)0.3, 0.05)$ & $G(100, (2)0.3, 0.1)$ & $G(100, (2)0.3, 0.15)$\\
        (kernel)& Opt. parameter, ARI    & Opt. parameter, ARI   & Opt. parameter, ARI\\
		\midrule
		pWalk	& 0.93,\;\;	1.00	& 0.87,\;\;	0.91	& 0.73,\;\;	0.66\\
		Walk	& 0.93,\;\;	1.00	& 0.67,\;\;	0.91	& 0.70,\;\;	0.65\\
		For		& 0.60,\;\;	0.99	& 0.97,\;\;	0.51	& 0.40,\;\;	0.01\\
		logFor	& 0.70,\;\;	1.00	& 0.40,\;\;	0.93	& 0.10,\;\;	0.68\\
		Comm	& 0.33,\;\;	1.00	& 0.33,\;\;	0.98	& 0.30,\;\;	0.77\\
		logComm	& 0.33,\;\;	1.00	& 0.47,\;\;	$\bm{1.00}$	& 0.57,\;\;	$\bm{0.91}$\\
		Heat	& 0.37,\;\;	1.00	& 0.60,\;\;	0.87	& 0.73,\;\;	0.15\\
		logHeat	& 0.37,\;\;	1.00	& 0.53,\;\;	0.99	& 0.37,\;\;	0.80\\
		SCT		& 0.40,\;\;	1.00	& 0.57,\;\;	0.94	& 0.43,\;\;	0.72\\
		SCCT	& 0.03,\;\;	1.00	& 0.57,\;\;	0.98	& 0.63,\;\;	0.80\\
		RSP		& 0.97,\;\;	1.00	& 0.97,\;\;	0.93	& 0.97,\;\;	0.67\\
		FE		& 0.90,\;\;	1.00	& 0.90,\;\;	0.91	& 0.87,\;\;	0.68\\
		SP-CT	& 0.00,\;\;	0.99	& 0.03,\;\;	0.78	& 0.07,\;\;	0.49\\
		\bottomrule
	\end{tabular}\caption{\label{t_Optt}Optimal family parameters and the corresponding ARI's}
\end{table}

Ошибка была в том, что подобранные параметры из таблицы выше принадлежат к диапазону $[0, 1]$, а значит их нужно преобразовывать к диапазону, специфичному для конкретной метрики. Я же этого не делал.
Вторая ошибка состояла в том, что я использовал тут близости вместо расстояний. Еще тогда, когда я строил их в прошлый раз, я заметил, что по близостям logComm совсем не обгоняет остальные меры, но по расстояниям эффект выраженный. Тут его тоже видно:

\begin{figure}[H] %tb
	\begin{minipage}{.24\textwidth} %49
		\leftfigure{\includegraphics[width=\linewidth]{pictures/4_pWalk.png}}
		\\\centerline{(a) pWalk}
	\end{minipage}
	\begin{minipage}{.24\textwidth} %49
		\leftfigure{\includegraphics[width=\linewidth]{pictures/4_Walk.png}}
		\\\centerline{(b) Walk}
	\end{minipage}
	\begin{minipage}{.24\textwidth} %49
		\leftfigure{\includegraphics[width=\linewidth]{pictures/4_For.png}}
		\\\centerline{(c) For}
	\end{minipage}
	\begin{minipage}{.24\textwidth} %49
		\leftfigure{\includegraphics[width=\linewidth]{pictures/4_logFor.png}}
		\\\centerline{(d) logFor}
	\end{minipage}
    \\[6pt]
	\begin{minipage}{.24\textwidth} %49
		\leftfigure{\includegraphics[width=\linewidth]{pictures/4_Comm.png}}
		\\\centerline{(e) Comm}
	\end{minipage}
	\begin{minipage}{.24\textwidth} %49
		\leftfigure{\includegraphics[width=\linewidth]{pictures/4_logComm.png}}
		\\\centerline{(f) logComm}
	\end{minipage}
	\begin{minipage}{.24\textwidth} %49
		\leftfigure{\includegraphics[width=\linewidth]{pictures/4_Heat.png}}
		\\\centerline{(g) Heat}
	\end{minipage}
	\begin{minipage}{.24\textwidth} %49
		\leftfigure{\includegraphics[width=\linewidth]{pictures/4_logHeat.png}}
		\\\centerline{(h) logHeat}
	\end{minipage}
    \\[6pt]

	\begin{minipage}{.195\textwidth} %49
		\leftfigure{\includegraphics[width=\linewidth]{pictures/4_SCT.png}}
		\\\centerline{(i) SCT}
	\end{minipage}
	\begin{minipage}{.195\textwidth} %49
		\leftfigure{\includegraphics[width=\linewidth]{pictures/4_SCCT.png}}
		\\\centerline{(j) SCCT}
	\end{minipage}
	\begin{minipage}{.195\textwidth} %49
		\leftfigure{\includegraphics[width=\linewidth]{pictures/4_RSP.png}}
		\\\centerline{(k) RSP}
	\end{minipage}
	\begin{minipage}{.195\textwidth} %49
		\leftfigure{\includegraphics[width=\linewidth]{pictures/4_FE.png}}
		\\\centerline{(l) FE}
	\end{minipage}
	\begin{minipage}{.195\textwidth} %49
		\leftfigure{\includegraphics[width=\linewidth]{pictures/4_SPCT.png}}
		\\\centerline{(m) SP-CT}
	\end{minipage}

    \caption{\label{f_Reject}Reject curves for the graph measures under study}
\end{figure}

\begin{figure}[H] %tb
	\begin{minipage}{.56\textwidth}
		\leftfigure{\includegraphics[width=.75\linewidth]{pictures/4_all.png}}
		\\\centerline{(a) All families}
	\end{minipage}%
	\begin{minipage}{.56\textwidth}
		\leftfigure{\includegraphics[width=.75\linewidth]{pictures/4_4best.png}}
		\\\centerline{(a) All families}
	\end{minipage}%
\caption{\label{f_Rcur}Average reject curves}
\end{figure}

На картинках выше видно, что Comm и logComm ведут себя довольно странно. Я вспомнил, что только из расстояний Comm и logComm мы берем корень. Я не могу вспомнить, чем все-таки он здесь оправдан, но это было нужно для того, чтобы результаты совпадали со статьей Studying new classes of graph metrics. Если убрать корень, то результаты становятся очень похожи на то, что было в "Do Logarithmic Proximity Measures Outperform Plain Ones in Graph Clustering?":

\begin{figure}[H] %tb
	\begin{minipage}{.33\textwidth}
		\leftfigure{\includegraphics[width=.75\linewidth]{pictures/4_Comm_alt.png}}
		\\\centerline{(a) All families}
	\end{minipage}%
	\begin{minipage}{.33\textwidth}
		\leftfigure{\includegraphics[width=.75\linewidth]{pictures/4_logComm_alt.png}}
		\\\centerline{(a) All families}
	\end{minipage}%
\caption{\label{f_Rcur}Alternative Comm and logComm}
\end{figure}

Тогда общая картина будет выглядеть так:

\begin{figure}[H] %tb
	\begin{minipage}{.56\textwidth}
		\leftfigure{\includegraphics[width=.75\linewidth]{pictures/4_all_alt.png}}
		\\\centerline{(a) All families}
	\end{minipage}%
	\begin{minipage}{.56\textwidth}
		\leftfigure{\includegraphics[width=.75\linewidth]{pictures/4_4best_alt.png}}
		\\\centerline{(a) All families}
	\end{minipage}%
\caption{\label{f_Rcur}Average reject curves with alternative Comm and logComm}
\end{figure}

В принципе, еще можно подозревать внешний вид графика pWalk. У меня есть подозрение, почему он такой: мы фиксируем параметр в [0, 1] и для каждого графа преобразуем в параметр меры в зависимости от спектрального радиуса матрицы $A$ ($ param = t / \rho(A)), t \in [0, 1]$). Могу проверить это, но, кажется, это не очень важно.


\newpage
\section{Graphs with classes of different sizes}
\begin{figure}[H]
	\begin{minipage}{.5\textwidth}
		\leftfigure{\includegraphics[width=.9\linewidth]{pictures/5_best1.png}} %0.8
		\\\centerline{(a) All families}
	\end{minipage}%
	\begin{minipage}{.5\textwidth}
		\leftfigure{\includegraphics[width=.9\linewidth]{pictures/5_best2.png}} %1.07
		\\\centerline{(b) Leading families}
	\end{minipage}
\caption{\label{f_difClas}Graphs with two classes of different sizes: clustering with optimal parameter values}
\end{figure}

\begin{figure}[H]
	\leftfigure{\includegraphics[width=.45\linewidth]{pictures/5_avg.png}}
\caption{\label{f_difClas1}Graphs with two classes of different sizes: random parameter values}
\end{figure}

\begin{figure}[H]
\samenumber
\begin{minipage}{.45\textwidth}
{\normalsize
$$
P=\begin{pmatrix}
    0.30& 0.20& 0.10& 0.15& 0.07& 0.25\\
    0.20& 0.24& 0.08& 0.13& 0.05& 0.17\\
    0.10& 0.08& 0.16& 0.09& 0.04& 0.12\\
    0.15& 0.13& 0.09& 0.20& 0.02& 0.14\\
    0.07& 0.05& 0.04& 0.02& 0.12& 0.04\\
    0.25& 0.17& 0.12& 0.14& 0.04& 0.40\\
  \end{pmatrix}.
$$}
\end{minipage}
\begin{minipage}{.45\textwidth}
	\leftfigure{\includegraphics[width=.85\linewidth]{{pictures/5_six}.png}}
\end{minipage}
\twocaptionwidth{.45\textwidth}{.45\textwidth}\phantom{\rightcaption{}}\rightcaption{\label{f_6classes}ARI of various measure families on a structure with 6 classes}
\end{figure}


\newpage
\section{Cluster analysis on several classical datasets}

Здесь ошибка была в том, что я зафиксировал число классов -- 2, хотя в датасете football их 12.
Теперь все похоже на статью:

\begin{figure}[H]
	\begin{minipage}{.32\textwidth}
		\leftfigure{\includegraphics[width=\linewidth]{pictures/6_football.png}}
		\\\centerline{(a) football}
	\end{minipage}
	\begin{minipage}{.32\textwidth}
		\leftfigure{\includegraphics[width=\linewidth]{pictures/6_polbooks.png}}
		\\\centerline{(b) polbooks}
	\end{minipage}
	\begin{minipage}{.32\textwidth}
		\leftfigure{\includegraphics[width=\linewidth]{pictures/6_zachary.png}}
		\\\centerline{(c) Zachary}
	\end{minipage}
    \\[10pt]
	\begin{minipage}{.32\textwidth}
		\leftfigure{\includegraphics[width=\linewidth]{pictures/6_news_2cl_1.png}}
		\\\centerline{(d) news\_2cl\_1}
	\end{minipage}
	\begin{minipage}{.32\textwidth}
		\leftfigure{\includegraphics[width=\linewidth]{pictures/6_news_2cl_2.png}}
		\\\centerline{(e) news\_2cl\_2}
	\end{minipage}
	\begin{minipage}{.32\textwidth}
		\leftfigure{\includegraphics[width=\linewidth]{pictures/6_news_2cl_3.png}}
		\\\centerline{(f) news\_2cl\_3}
	\end{minipage}
	\\[10pt]
    \begin{minipage}{\textwidth}
        \hfill\includegraphics[width=0.7\linewidth]{pictures/6_legend.png}\hfill
	\end{minipage}
  \caption{\label{f_datasets}ARI of various measure families on classical datasets}
\end{figure}

\end{document}
