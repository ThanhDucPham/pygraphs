\documentclass{article}
\usepackage[T2A]{fontenc}
\usepackage[utf8]{inputenc}
\usepackage[english, russian]{babel}

\usepackage{geometry}
\geometry{a4paper, top=1cm, bottom=2cm, left=1cm, right=1cm}

\usepackage{indentfirst}
\usepackage{amssymb,amsfonts,amsmath}
\usepackage{cite,enumerate}
\usepackage[pdftex,colorlinks,unicode,bookmarks]{hyperref}

\usepackage{floatrow}
\usepackage{caption}
\usepackage{subcaption}
\usepackage{multirow}
\usepackage[table]{xcolor}
\usepackage{graphicx}
\usepackage{booktabs}
\usepackage{bm}

\title{
        Воспроизведение результатов статьи в \href{https://github.com/illusionww/pygraphs}{pygraphs}.
}
\author{Владимир Ивашкин}

\begin{document}

\maketitle

\setcounter{section}{1}

\section{Logarithmic vs. plain measures}

Не ясно, в оригинале был RI или ARI. Если был ARI, то он на тот момент был неправильным. Привожу тут оба варианта

\begin{figure}[H]
	\includegraphics[width=.99\linewidth]{pictures/2_1_RI.png}
	\includegraphics[width=.99\linewidth]{pictures/2_1_ARI.png}
	\caption{\label{f_2_1} $G(100, (2)0.2, 0.05)$, RI and ARI respectively}
\end{figure}

\begin{figure}[H]
	\includegraphics[width=.99\linewidth]{pictures/2_2_RI.png}
	\includegraphics[width=.99\linewidth]{pictures/2_2_ARI.png}
	\caption{\label{f_2_2} $G(100, (3)0.3, 0.1)$, RI and ARI respectively}
\end{figure}


\begin{figure}[H]
	\includegraphics[width=.99\linewidth]{pictures/2_3_RI.png}
	\includegraphics[width=.99\linewidth]{pictures/2_3_ARI.png}
	\caption{\label{f_2_3} $G(200, (2)0.3, 0.1)$, RI and ARI respectively}
\end{figure}

\nopagebreak


\newpage
\section{Competition by Copeland's score}
\begin{table}[H]{\small
	\centering
	\begin{tabular}{lrrrrrrrrr}
		\toprule
\textbf{Nodes}   & 100&  100& 100&  100& 200&  200& 200&  200&{\textbf{Sum}} \\
\textbf{Classes} &   2&    2&   4&    4&   2&    2&   4&    4&{\textbf{  of}} \\
$p_{out}$        & 0.1&	0.15& 0.1& 0.15& 0.1& 0.15& 0.1& 0.15&{\textbf{scores}} \\
		\midrule
logComm  & 10 & 512 & 406 & -122 & 580 & 333 & 152 & 600 & $\bm{2471}$\\
Comm & 4 & 185 & 86 & 448 & 244 & 297 & 442 & 246 & $\bm{1952}$\\
SCCT & 10 & 287 & 188 & 148 & 289 & 238 & 76 & 458 & $\bm{1694}$\\
Heat & 10 & -310 & 86 & 448 & 136 & 332 & 442 & -260 & $\bm{884}$\\
pWalk & -3 & -41 & 86 & 448 & -41 & -106 & 442 & -138 & $\bm{647}$\\
logHeat & 4 & 67 & -16 & -294 & 202 & 332 & -292 & 166 & $\bm{169}$\\
SCT & -6 & 51 & -106 & 148 & -39 & 69 & 76 & -42 & $\bm{151}$\\
logFor & -8 & 33 & -70 & -298 & 3 & -83 & -262 & 50 & $\bm{-635}$\\
FE & 0 & -12 & -104 & -294 & -97 & -102 & -294 & -4 & $\bm{-907}$\\
For & -10 & -560 & 86 & 448 & -568 & -546 & 442 & -260 & $\bm{-968}$\\
RSP & -3 & 92 & -132 & -358 & -107 & -1 & -336 & -124 & $\bm{-969}$\\
Walk & 4 & 20 & -40 & -316 & -144 & -221 & -346 & -98 & $\bm{-1141}$\\
SP-CT & -12 & -324 & -470 & -406 & -458 & -542 & -542 & -594 & $\bm{-3348}$\\
		\bottomrule
	\end{tabular}
	\caption{\label{t_CopComp} Optimal parameters}
}\end{table}

\begin{table}[H]{\small
	\centering
	\begin{tabular}{lrrrrrrrrr}
		\toprule
\textbf{Nodes}   & 100&  100& 100&  100& 200&  200& 200&  200&{\textbf{Sum}}\\
\textbf{Classes} &   2&    2&   4&    4&   2&    2&   4&    4&{\textbf{  of}}\\
$p_{out}$        & 0.1&	0.15& 0.1& 0.15& 0.1& 0.15& 0.1& 0.15&{\textbf{scores}}\\
		\midrule
logComm & 440 & 501 & 466 & 340 & 398 & 565 & 574 & 582 & $\bm{3866}$\\
SCCT & 263 & 295 & 360 & 184 & 295 & 397 & 438 & 370 & $\bm{2602}$\\
Comm & 109 & 149 & 106 & 120 & 198 & 60 & 168 & 158 & $\bm{1068}$\\
logHeat & 236 & 59 & 80 & 32 & 391 & 11 & 148 & 98 & $\bm{1055}$\\
logFor & -23 & 57 & 148 & 116 & -126 & 44 & 134 & 94 & $\bm{444}$\\
FE & -74 & 80 & 50 & 120 & -30 & 30 & 38 & 52 & $\bm{266}$\\
Walk & -79 & 119 & 114 & 102 & -84 & -4 & 20 & 76 & $\bm{264}$\\
SCT & -27 & 27 & 4 & -32 & 52 & -6 & 36 & 30 & $\bm{84}$\\
pWalk & 45 & 1 & 20 & 10 & -62 & -31 & -10 & 26 &$\bm{-1}$\\
Heat & 296 & -322 & -492 & -445 & 386 & 249 & -215 & -472 & $\bm{-1015}$\\
RSP & -313 & -117 & -16 & 14 & -338 & -268 & -280 & -84 & $\bm{-1402}$\\
SP-CT & -482 & -287 & -250 & 0 & -585 & -460 & -452 & -352 & $\bm{-2868}$\\
For & -391 & -562 & -590 & -561 & -495 & -587 & -599 & -578 & $\bm{-4363}$\\
		\bottomrule
	\end{tabular}
	\caption{\label{t_CopComp} 90th percentiles}
}\end{table}

\newpage
\section{Reject curves}



\begin{figure}[H]
	\includegraphics[width=.9\linewidth]{pictures/4_01.png}
	\caption{\label{f_4_big} Reject curves for the graph measures under study}
\end{figure}

\begin{table}[H]
	\begin{tabular}{lrrrr}
			\toprule
Measure & $p_{out}=$ 0.5 & 0.1 & 0.15\\
		\midrule
pWalk & 0.86 & 0.80 & 0.86\\
Walk & 0.82 & 0.76 & 0.76\\
For & 0.96 & 0.98 & 0.44\\
logFor & 0.72 & 0.40 & 0.28\\
Comm & 0.42 & 0.36 & 0.24\\
logComm & 0.46 & 0.54 & 0.64\\
Heat & 0.70 & 0.74 & 0.82\\
logHeat & 0.70 & 0.46 & 0.18\\
SCT & 0.46 & 0.50 & 0.48\\
SCCT & 0.98 & 0.74 & 0.44\\
RSP & 0.98 & 0.98 & 0.98\\
FE & 0.96 & 0.92 & 0.76\\
SP-CT & 0.00 & 0.04 & 0.36\\
		\bottomrule
	\end{tabular}\caption{\label{t_4_opt} Optimal family parameters for $G(100, (2)0.3, p_{out})$}
\end{table}

\begin{figure}[H]
	\includegraphics[width=.37\linewidth]{pictures/4_bigpicture.png}
	%\caption{\label{f_4_big} Reject curves for the graph measures under study}
\end{figure}

\begin{figure}[H]
	\begin{minipage}{.49\textwidth}
		\includegraphics[width=\linewidth]{pictures/4_all.png}\\
		\centerline{(a) All families}
	\end{minipage}
	\begin{minipage}{.49\textwidth}
		\includegraphics[width=\linewidth]{pictures/4_4best.png}\\
		\centerline{(b) Four best}
	\end{minipage}
\caption{\label{f_Rcur}Average reject curves}
\end{figure}

\newpage
\section{Graphs with classes of different sizes}
\begin{figure}[H]
	\begin{minipage}{.49\textwidth}
		\includegraphics[width=\linewidth]{pictures/5_best1.png}\\
		\centerline{(a) All families}
	\end{minipage}%
	\begin{minipage}{.49\textwidth}
		\includegraphics[width=\linewidth]{pictures/5_best2.png}\\
		\centerline{(b) Leading families}
	\end{minipage}
\caption{\label{f_difClas}Graphs with two classes of different sizes: clustering with optimal parameter values}
\end{figure}

\begin{figure}[H]
	\includegraphics[width=.45\linewidth]{pictures/5_avg.png}
	\caption{\label{f_difClas1}Graphs with two classes of different sizes: random parameter values}
\end{figure}

\begin{figure}[H]
	\begin{minipage}{.45\textwidth}
		{\normalsize
		$$
		P=\begin{pmatrix}
   			0.30& 0.20& 0.10& 0.15& 0.07& 0.25\\
   			0.20& 0.24& 0.08& 0.13& 0.05& 0.17\\
    		0.10& 0.08& 0.16& 0.09& 0.04& 0.12\\
    		0.15& 0.13& 0.09& 0.20& 0.02& 0.14\\
    		0.07& 0.05& 0.04& 0.02& 0.12& 0.04\\
    		0.25& 0.17& 0.12& 0.14& 0.04& 0.40\\
  		\end{pmatrix}.
		$$}
	\end{minipage}
	\begin{minipage}{.45\textwidth}
		\includegraphics[width=.85\linewidth]{{pictures/5_six}.png}
	\end{minipage}
	\caption{\label{f_6classes}ARI of various measure families on a structure with 6 classes}
\end{figure}


\newpage
\section{Cluster analysis on several classical datasets}

Здесь ошибка была в том, что я зафиксировал число классов -- 2, хотя в датасете football их 12.
Теперь все похоже на статью:

\begin{figure}[H]
	\begin{minipage}{.32\textwidth}
		\includegraphics[width=\linewidth]{pictures/6_football.png}
		\\\centerline{(a) football}
	\end{minipage}
	\begin{minipage}{.32\textwidth}
		\includegraphics[width=\linewidth]{pictures/6_polbooks.png}
		\\\centerline{(b) polbooks}
	\end{minipage}
	\begin{minipage}{.32\textwidth}
		\includegraphics[width=\linewidth]{pictures/6_zachary.png}
		\\\centerline{(c) zachary}
	\end{minipage}
    \\[10pt]
	\begin{minipage}{.32\textwidth}
		\includegraphics[width=\linewidth]{pictures/6_news_2cl_1.png}
		\\\centerline{(d) news\_2cl\_1}
	\end{minipage}
	\begin{minipage}{.32\textwidth}
		\includegraphics[width=\linewidth]{pictures/6_news_2cl_2.png}
		\\\centerline{(e) news\_2cl\_2}
	\end{minipage}
	\begin{minipage}{.32\textwidth}
		\includegraphics[width=\linewidth]{pictures/6_news_2cl_3.png}
		\\\centerline{(f) news\_2cl\_3}
	\end{minipage}
	\\[10pt]
    \begin{minipage}{\textwidth}
        \includegraphics[width=\linewidth]{pictures/6_legend.png}
	\end{minipage}
  \caption{\label{f_datasets}ARI of various measure families on classical datasets}
\end{figure}

\end{document}
